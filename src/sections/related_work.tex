Several projects have proposed solutions to the problem of autonomous drone landing by
marking landing pads with various fiducial markers (April Tag, ArUco, or custom markers).\cite{high_velocity_landing}\cite{visual_servoing}\cite{vision_based_x_platform}\cite{wynn}\cite{accurate_landing_UAV_ground_pattern}
Others use photographic matching between pictures of a landing site captured on takeoff.\cite{drone_landing_unstructured_environments}
The general trend in these methods is to first navigate near the anticipated landing site via GPS, and then to descend using visual clues to improve on the GPS accuracy,
with the correct assumption that GPS does not provide a position estimate with sufficiently high resolution to land on a small platform
(even with a clear view of the sky which provides reliable GPS reception).
%GPS is even less reliable in indoor environments or in canyons where reception is worse.
The drones contain one or two fixed cameras to identify markings or landmarks that provide a position estimate for the drone relative to the landing pad or landing site.
One main challenge in these methods is that,
since the camera is fixed to the drone body,
and since the drone modulates its orientation as its main means of positional control,
the movement of the drone affects the camera's field of view in way that can cause the marker to be lost.
Further, the markers can eclipse the camera's field of view once the drone gets too close, meaning that the drone may have to finish the landing blind.

Another method employs a more sophisticated workflow that avoids the need for fiducial markers.\cite{rooftop_landing}
The drone explores an area where a landing should take place, capturing images from a fixed, downward-facing camera.
The images contain tags specifying the location where they were captured, and then the drone streams them to a nearby
computer for analysis.
The system generates disparity maps and feature matches which imitate stereo image processing, and creates a 3D map of the terrain below.
Sufficiently flat and large areas serve as possible landing sites, and the system chooses one such site for the drone to land at.
The method is succesful in identifying viable landing sites, but requires a sophisticated ground control station and requires some adjustment before it can run in real time.
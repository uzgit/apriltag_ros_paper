Autonomous landing is one remaining major hurdle in fully autonomous multirotor drone flight.
Many other tasks such as takeoff, waypoint-to-waypoint flight, and in-flight tasks such as picture-taking have been
sufficiently automated.
However, landing is difficult and risky enough that no widespread, autonomous solution is currently available.
Landing with GPS alone is not reliable because target landing sites are typically small and GPS does not guarantee sufficient accuracy.
Some projects have proposed autonomous landing methods (see Section \ref{section:related_work})
that use some combination of different components such as RGB cameras, ultrasonic sensors,
sophisticated ground control stations, and fiducial markers.
These methods serve as a proof of concept for the idea of autonomous landing, but have not been widely adopted yet.
{\color{red}Additionally, autonomous landing algorithms typically require either the addition of non-standard components (complexifying the drone system),
or depend on a fixed, downward facing camera (limiting the range of behaviors during which the landing pad can be identified).}

\todo[inline]{How do we get from the mentioned methods to april tag? What defines the knowledge gap of \emph{this} paper? What are the properties of a solution to this problem? Why did we chose april tag?}

This project introduces a flexible landing system based on the April Tag\cite{apriltag3_paper} fiducial system.
All computation occurs onboard the drone in order to avoid the need for extra data transmission, latency, and a
sophisticated ground station.
The method is implemented as a set of ROS modules and can be constrained to a single computational board,
or distributed to multiple network-connected boards.
The method assumes a near-level landing pad marked with nested April Tag fiducial markers for reliable, well-tested
pose estimation that functions over a large range of distances.
An edited April Tag ROS module adds message attributes for this application and generates position targets
regardless of the orientation of the markers in the camera frame.
This gives the option to track the markers using a gimbal-mounted camera,
increasing the range and reliability of detection over methods that use a fixed, downward-facing camera.
The method also requires no sensor data on the orientation of the gimbal, which is helpful as many widely-available
gimbals do not provide this data.
%Finally, this method is able to abort a landing attempt during unacceptable situations,
%such as if the drone is blown off course by a gust of wind.

The ability to carry out many missions autonomously, and over long periods of time,
is essential to widespread integration of drones into industry operations.
Drones have already provided a cheaper, faster, and less risky alternative to humans in tasks such as
monitoring ship traffic in seaports\cite{security_integration},
geological surveys\cite{eit},
and infrastructure inspection.\cite{maritine_infrastructure_inspection}
However, they still require human operators for landing, charging, and maintenance at the least.
Autonomous landing is the key to further enabling fully-autonomous mission cycles requiring minimal attention from human operators.